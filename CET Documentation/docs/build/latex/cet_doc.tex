%% Generated by Sphinx.
\def\sphinxdocclass{report}
\documentclass[letterpaper,10pt,english]{sphinxmanual}
\ifdefined\pdfpxdimen
   \let\sphinxpxdimen\pdfpxdimen\else\newdimen\sphinxpxdimen
\fi \sphinxpxdimen=.75bp\relax

\PassOptionsToPackage{warn}{textcomp}
\usepackage[utf8]{inputenc}
\ifdefined\DeclareUnicodeCharacter
% support both utf8 and utf8x syntaxes
  \ifdefined\DeclareUnicodeCharacterAsOptional
    \def\sphinxDUC#1{\DeclareUnicodeCharacter{"#1}}
  \else
    \let\sphinxDUC\DeclareUnicodeCharacter
  \fi
  \sphinxDUC{00A0}{\nobreakspace}
  \sphinxDUC{2500}{\sphinxunichar{2500}}
  \sphinxDUC{2502}{\sphinxunichar{2502}}
  \sphinxDUC{2514}{\sphinxunichar{2514}}
  \sphinxDUC{251C}{\sphinxunichar{251C}}
  \sphinxDUC{2572}{\textbackslash}
\fi
\usepackage{cmap}
\usepackage[T1]{fontenc}
\usepackage{amsmath,amssymb,amstext}
\usepackage{babel}



\usepackage{times}
\expandafter\ifx\csname T@LGR\endcsname\relax
\else
% LGR was declared as font encoding
  \substitutefont{LGR}{\rmdefault}{cmr}
  \substitutefont{LGR}{\sfdefault}{cmss}
  \substitutefont{LGR}{\ttdefault}{cmtt}
\fi
\expandafter\ifx\csname T@X2\endcsname\relax
  \expandafter\ifx\csname T@T2A\endcsname\relax
  \else
  % T2A was declared as font encoding
    \substitutefont{T2A}{\rmdefault}{cmr}
    \substitutefont{T2A}{\sfdefault}{cmss}
    \substitutefont{T2A}{\ttdefault}{cmtt}
  \fi
\else
% X2 was declared as font encoding
  \substitutefont{X2}{\rmdefault}{cmr}
  \substitutefont{X2}{\sfdefault}{cmss}
  \substitutefont{X2}{\ttdefault}{cmtt}
\fi


\usepackage[Bjarne]{fncychap}
\usepackage{sphinx}

\fvset{fontsize=\small}
\usepackage{geometry}


% Include hyperref last.
\usepackage{hyperref}
% Fix anchor placement for figures with captions.
\usepackage{hypcap}% it must be loaded after hyperref.
% Set up styles of URL: it should be placed after hyperref.
\urlstyle{same}


\usepackage{sphinxmessages}
\setcounter{tocdepth}{1}



\title{CET\_doc}
\date{Feb 17, 2021}
\release{-}
\author{Noel Strahm}
\newcommand{\sphinxlogo}{\vbox{}}
\renewcommand{\releasename}{Release}
\makeindex
\begin{document}

\pagestyle{empty}
\sphinxmaketitle
\pagestyle{plain}
\sphinxtableofcontents
\pagestyle{normal}
\phantomsection\label{\detokenize{index::doc}}


\noindent{\hspace*{\fill}\sphinxincludegraphics[width=500\sphinxpxdimen,height=293\sphinxpxdimen]{{CET_bild}.png}\hspace*{\fill}}


\chapter{Homepage}
\label{\detokenize{index:homepage}}
Carbon\sphinxhyphen{}Emission\sphinxhyphen{}task.ch


\chapter{About}
\label{\detokenize{index:about}}\begin{itemize}
\item {} 
The Carbon Emission Task (CET) as a new paradigm grounded in experimental economics to assess actual pro\sphinxhyphen{}environmental behavior

\item {} 
The CET can be used in laboratory, online, or classroom studies.

\item {} 
Participants face repeated tradeoffs between financial bonus opportunities paired with real carbon emissions and foregoing such opportunities while staying carbon\sphinxhyphen{}neutral.

\end{itemize}


\chapter{Support}
\label{\detokenize{index:support}}
For help, please contact \sphinxhref{mailto:someone@iop.unibe.ch}{someone@iop.unibe.ch}


\chapter{Contents:}
\label{\detokenize{index:contents}}



\section{Set up the CET in oTree}
\label{\detokenize{setup_in_otree:set-up-the-cet-in-otree}}\label{\detokenize{setup_in_otree:setup-in-otree}}\label{\detokenize{setup_in_otree::doc}}
This is how you import the Carbon Emission Task in your oTree framework.


\subsection{Step 1: Download the CET}
\label{\detokenize{setup_in_otree:step-1-download-the-cet}}
Download the CET from Github: (Insert Download Link here)


\subsection{Step 2: Add the CET to your oTree Project}
\label{\detokenize{setup_in_otree:step-2-add-the-cet-to-your-otree-project}}
Copy the CET folder into your oTree project directory like this:
\begin{description}
\item[{— Your\_oTree\_Project}] \leavevmode
\begin{DUlineblock}{0em}
\item[] \sphinxhyphen{} otree\_app1
\item[] \sphinxhyphen{} otree\_app2
\item[] ..
\item[] \sphinxhyphen{} CET
\item[] \sphinxhyphen{} manage.py
\item[] \sphinxhyphen{} settings.py
\end{DUlineblock}

\end{description}


\subsection{Step 3: Change session configurations in settings.py}
\label{\detokenize{setup_in_otree:step-3-change-session-configurations-in-settings-py}}
Include the CET app in your settings.py file. To do this append the following code in the
SESSION\_CONFIGS list.

\begin{sphinxVerbatim}[commandchars=\\\{\}]
\PYG{n+nb}{dict}\PYG{p}{(}
    \PYG{n}{name}\PYG{o}{=}\PYG{l+s+s1}{\PYGZsq{}}\PYG{l+s+s1}{cet}\PYG{l+s+s1}{\PYGZsq{}}\PYG{p}{,}
    \PYG{n}{display\PYGZus{}name}\PYG{o}{=}\PYG{l+s+s2}{\PYGZdq{}}\PYG{l+s+s2}{CET Light Version}\PYG{l+s+s2}{\PYGZdq{}}\PYG{p}{,}
    \PYG{n}{num\PYGZus{}demo\PYGZus{}participants}\PYG{o}{=}\PYG{l+m+mi}{40}\PYG{p}{,}
    \PYG{n}{app\PYGZus{}sequence}\PYG{o}{=}\PYG{p}{[}\PYG{l+s+s1}{\PYGZsq{}}\PYG{l+s+s1}{cet}\PYG{l+s+s1}{\PYGZsq{}}\PYG{p}{]}\PYG{p}{,}
\PYG{p}{)}
\end{sphinxVerbatim}

The {\hyperref[\detokenize{cet_data_csv:csv-file}]{\sphinxcrossref{\DUrole{std,std-ref}{CET Data (csv file)}}}} provides the values for a player’s bonus. The currency used is the Great British Pound. If you want
to use a different currency, the data in the cet\_data file has to be adjusted accordingly. If no changes are made to
the cet\_data file the currency parameters have to be specified as such:

\begin{sphinxVerbatim}[commandchars=\\\{\}]
\PYG{c+c1}{\PYGZsh{} e.g. EUR, GBP, CNY, JPY}
\PYG{n}{REAL\PYGZus{}WORLD\PYGZus{}CURRENCY\PYGZus{}CODE} \PYG{o}{=} \PYG{l+s+s1}{\PYGZsq{}}\PYG{l+s+s1}{GBP}\PYG{l+s+s1}{\PYGZsq{}}
\PYG{n}{USE\PYGZus{}POINTS} \PYG{o}{=} \PYG{k+kc}{False}
\end{sphinxVerbatim}




\section{CET Data (csv file)}
\label{\detokenize{cet_data_csv:cet-data-csv-file}}\label{\detokenize{cet_data_csv:csv-file}}\label{\detokenize{cet_data_csv::doc}}
The cet\_data.csv file provides the player’s {\hyperref[\detokenize{Player_fields:carbon-ref}]{\sphinxcrossref{\DUrole{std,std-ref}{Carbon}}}}, {\hyperref[\detokenize{Player_fields:car-miles-ref}]{\sphinxcrossref{\DUrole{std,std-ref}{Car Miles}}}} and {\hyperref[\detokenize{Player_fields:bonus-ref}]{\sphinxcrossref{\DUrole{std,std-ref}{Bonus}}}} values.
There are 20 unique combinations of carbon, car miles and bonus values in the file. Those rows are repeated once,
yielding a total of 40 data rows. Each data row supplies data for a CET round, hence the number of data rows is
equal to the number of rounds in the CET (40).

Experimenters can extend the csv file with more data rows, which in turn increases the number of rounds in the CET.
Furthermore, the data values can be modified, if experimenters want to explore different framework conditions
for the CET. For example a different currency other than GBP, different carbon values, higher or lower bonuses or different analogies other than car miles
to show the carbon footprint of each decision.

Moreover, some modifications can be made in the Constants class of the models.py file.




\section{Constants}
\label{\detokenize{Constants:constants}}\label{\detokenize{Constants:id1}}\label{\detokenize{Constants::doc}}
The following Constants can be set in the Constants class:


\subsection{Random Payoff}
\label{\detokenize{Constants:random-payoff}}\label{\detokenize{Constants:id2}}\begin{itemize}
\item {} 
\sphinxcode{\sphinxupquote{random\_payoff}} is used to define your preferred payoff calculation for the \sphinxcode{\sphinxupquote{player.payoff}} field

\item {} 
If set to \sphinxcode{\sphinxupquote{True}} a player’s payoff is only calculated in the \sphinxcode{\sphinxupquote{paying\_round}}. The payoff in all other rounds is 0. See {\hyperref[\detokenize{Subsession_fields:paying-round-ref}]{\sphinxcrossref{\DUrole{std,std-ref}{Paying Round}}}}

\item {} 
If set to \sphinxcode{\sphinxupquote{False}} a player’s payoff is calculated in all rounds.

\end{itemize}


\subsection{Random Saved Emission}
\label{\detokenize{Constants:random-saved-emission}}\label{\detokenize{Constants:random-emission}}\begin{itemize}
\item {} 
\sphinxcode{\sphinxupquote{random\_saved\_emission}} is used to define how the \sphinxcode{\sphinxupquote{Subsession.sum\_saved\_emission}} field is calculated. See {\hyperref[\detokenize{Subsession_fields:sum-saved-emission-ref}]{\sphinxcrossref{\DUrole{std,std-ref}{Sum saved Emission}}}}

\item {} 
If set to \sphinxcode{\sphinxupquote{True}} the sum is calculated by adding each player’s saved emission in the \sphinxcode{\sphinxupquote{paying\_round}}

\item {} 
If set to \sphinxcode{\sphinxupquote{False}} the saved emission for all rounds is added.

\end{itemize}


\subsection{Bot Criteria}
\label{\detokenize{Constants:bot-criteria}}\label{\detokenize{Constants:bot-criteria-ref}}\begin{itemize}
\item {} 
\sphinxcode{\sphinxupquote{Bot\_criteria}} is used to define the criteria as to when a player is seen as a bot.

\item {} 
\sphinxcode{\sphinxupquote{Bot\_criteria}} must be a value between 0\sphinxhyphen{}1 that represents the percentage of autonomous decisions a player has to make in order to be seen as human.

\item {} 
Example: The definition below means a player must decide autonomously (No Timeout happened) in at least 75\% of all rounds. Otherwise the player is regarded as a bot.

\end{itemize}

\begin{sphinxVerbatim}[commandchars=\\\{\}]
\PYG{n}{Bot\PYGZus{}criteria} \PYG{o}{=} \PYG{l+m+mf}{0.75}
\end{sphinxVerbatim}


\subsection{Num Rounds}
\label{\detokenize{Constants:num-rounds}}\begin{itemize}
\item {} 
This is an oTree specific variable that defines the number of rounds of the CET.

\item {} 
The number of rounds is defined as the length of the {\hyperref[\detokenize{cet_data_csv:csv-file}]{\sphinxcrossref{\DUrole{std,std-ref}{CET Data (csv file)}}}} that contains the carbon data.

\end{itemize}




\section{Subsession Class}
\label{\detokenize{Subsession_fields:subsession-class}}\label{\detokenize{Subsession_fields:subsession-fields}}\label{\detokenize{Subsession_fields::doc}}
Here’s an overview of all fields and functions in the Subsession class:


\subsection{creating\_session()}
\label{\detokenize{Subsession_fields:creating-session}}\label{\detokenize{Subsession_fields:creating-session-ref}}\begin{itemize}
\item {} 
The \sphinxcode{\sphinxupquote{paying\_round}} is set as a random round between 1 and \sphinxcode{\sphinxupquote{num\_rounds}}.

\item {} 
The order of questions is randomized for each player

\item {} 
The \sphinxcode{\sphinxupquote{payer.carbon}}, \sphinxcode{\sphinxupquote{player.car\_miles}} and \sphinxcode{\sphinxupquote{player.bonus}} fields are initialized for all rounds.

\end{itemize}


\subsection{Paying Round}
\label{\detokenize{Subsession_fields:paying-round}}\label{\detokenize{Subsession_fields:paying-round-ref}}\begin{itemize}
\item {} 
The \sphinxcode{\sphinxupquote{paying\_round}} is the round where the player’s payoff is calculated if \sphinxcode{\sphinxupquote{random\_payoff = True}}. See {\hyperref[\detokenize{Constants:random-payoff}]{\sphinxcrossref{\DUrole{std,std-ref}{Random Payoff}}}}

\item {} 
A player’s saved emission in the \sphinxcode{\sphinxupquote{paying\_round}} is added to the \sphinxcode{\sphinxupquote{sum\_saved\_emission}} field if \sphinxcode{\sphinxupquote{random\_saved\_emission = True}}. See {\hyperref[\detokenize{Constants:random-emission}]{\sphinxcrossref{\DUrole{std,std-ref}{Random Saved Emission}}}}

\end{itemize}


\subsection{Sum saved Emission}
\label{\detokenize{Subsession_fields:sum-saved-emission}}\label{\detokenize{Subsession_fields:sum-saved-emission-ref}}

\subsubsection{Field}
\label{\detokenize{Subsession_fields:field}}\begin{itemize}
\item {} 
The \sphinxcode{\sphinxupquote{sum\_saved\_emission}} field is the sum of the \sphinxcode{\sphinxupquote{saved\_emission}} player field for all players.

\item {} 
The sum is either calculated across the \sphinxcode{\sphinxupquote{paying\_round}} or across all rounds depending on the {\hyperref[\detokenize{Constants:random-emission}]{\sphinxcrossref{\DUrole{std,std-ref}{Random Saved Emission}}}} field.

\item {} 
The field is used as an input in the \sphinxcode{\sphinxupquote{send\_payment\_mail()}} function

\item {} 
Excludes all players that ar seen as bots.

\end{itemize}


\subsubsection{Function}
\label{\detokenize{Subsession_fields:function}}\begin{itemize}
\item {} 
The \sphinxcode{\sphinxupquote{set\_sum\_saved\_emission()}} function sets the \sphinxcode{\sphinxupquote{sum\_saved\_emission}} field.

\item {} 
Checks if a player is a bot. See {\hyperref[\detokenize{Constants:bot-criteria-ref}]{\sphinxcrossref{\DUrole{std,std-ref}{Bot Criteria}}}}

\item {} 
If a player is not a bot then the total \sphinxcode{\sphinxupquote{saved\_emission}} of all players is added to the \sphinxcode{\sphinxupquote{sum\_saved\_emission}} (Either across the paying round or all rounds)

\item {} 
A player has to finish all rounds of the CET, so that the correct data is available. Therefore, this function is only called in the last round of the CET. See {\hyperref[\detokenize{pages:exp-page}]{\sphinxcrossref{\DUrole{std,std-ref}{Experiment Page}}}}

\end{itemize}


\subsection{All Players Finished}
\label{\detokenize{Subsession_fields:all-players-finished}}

\subsubsection{Field}
\label{\detokenize{Subsession_fields:id1}}\begin{itemize}
\item {} 
\sphinxcode{\sphinxupquote{all\_players\_finished}} is a Boolean field that turns True once all players have finished the CET.

\end{itemize}


\subsubsection{Function}
\label{\detokenize{Subsession_fields:id2}}\begin{itemize}
\item {} 
\sphinxcode{\sphinxupquote{set\_all\_players\_finished()}} calculates how many players in total have finished the CET (\sphinxcode{\sphinxupquote{sum\_finished}}).

\item {} 
If \sphinxcode{\sphinxupquote{sum\_finished}} = Number of participants then the \sphinxcode{\sphinxupquote{all\_players\_finished}} field turns \sphinxcode{\sphinxupquote{True}}.

\item {} 
This function is only called once (for each player): When the player hits the “next” button on the {\hyperref[\detokenize{pages:results}]{\sphinxcrossref{\DUrole{std,std-ref}{Results Page}}}}

\end{itemize}


\subsection{Helpful prints}
\label{\detokenize{Subsession_fields:helpful-prints}}\begin{itemize}
\item {} 
The \sphinxcode{\sphinxupquote{helpful\_prints()}} function prints helpful information about the current state of many player and subsession fields to the terminal.

\item {} 
The function can be extended at will and be used for bug fixing purposes, if a new field is added.

\item {} 
The function is called in every round of the CET and when a player finishes the CET.

\end{itemize}


\subsection{Send Payment Mail}
\label{\detokenize{Subsession_fields:send-payment-mail}}
The \sphinxcode{\sphinxupquote{send\_payment\_mail()}} function is used to automate carbon\sphinxhyphen{}emission certificate purchases for experiments with real\sphinxhyphen{}carbon externalities, such as the CET.


\subsubsection{Requirements}
\label{\detokenize{Subsession_fields:requirements}}\begin{itemize}
\item {} 
The modules requests and smtplib have to be imported at the top of the models.py file.

\item {} 
A valid account from an SMTP service provider is needed. The credentials of the accout have to be specified at
the top of the function.

\end{itemize}


\subsubsection{Parameters}
\label{\detokenize{Subsession_fields:parameters}}\begin{itemize}
\item {} 
\sphinxcode{\sphinxupquote{weight\_to\_donate}}: A float value used to pass the amount of carbon emission that is saved by the experimental participants. The {\hyperref[\detokenize{Subsession_fields:sum-saved-emission-ref}]{\sphinxcrossref{\DUrole{std,std-ref}{Sum saved Emission}}}} is used for this.

\item {} 
\sphinxcode{\sphinxupquote{unit}}: A string value that defines the unit of the saved carbon emission. The following values are accepted: \sphinxcode{\sphinxupquote{{[}"mg", "g", "kg", "t", "oz", "lbs", "st"{]}}}

\item {} 
\sphinxcode{\sphinxupquote{experiment\_name}}: A string value that specifies the name of the experiment (e.g. “Carbon Emission Task”)

\item {} 
\sphinxcode{\sphinxupquote{payment\_e\_mail\_name}}: A string that specifies the name of the person or team that receives the mail

\item {} 
\sphinxcode{\sphinxupquote{payment\_e\_mail\_to}}: A list containing the mail addresses of all recipients . If the mail is only to be sent to one address then a single string can be passed to the function.

\end{itemize}


\subsubsection{How it works (basic)}
\label{\detokenize{Subsession_fields:how-it-works-basic}}\begin{enumerate}
\sphinxsetlistlabels{\arabic}{enumi}{enumii}{}{.}%
\item {} 
The \sphinxcode{\sphinxupquote{weight\_to\_donate}} value is converted to metric tons. The conversion is based on the \sphinxcode{\sphinxupquote{unit}} value.

\item {} 
The current CO2 price per ton for emission certificates is fetched from a webservice provided by \sphinxstyleemphasis{Compensators}

\item {} 
The price of the carbon\sphinxhyphen{}emission certificate is calculated.

\item {} 
A mail is sent to all addresses within the \sphinxcode{\sphinxupquote{payment\_e\_mail\_to}} list.
The mail includes the total weight of carbon\sphinxhyphen{}emission saved,
the current price per ton for carbon\sphinxhyphen{}emission certificates, as well as the link to \sphinxstyleemphasis{Compensators} donation form
to make the carbon\sphinxhyphen{}emission certificate purchase. These contents can be changed at will.

\end{enumerate}


\subsubsection{When is send\_payment\_mail() called?}
\label{\detokenize{Subsession_fields:when-is-send-payment-mail-called}}\label{\detokenize{Subsession_fields:mail-call-ref}}\begin{itemize}
\item {} 
The function is called when all players have finished the CET. For an example see {\hyperref[\detokenize{pages:results}]{\sphinxcrossref{\DUrole{std,std-ref}{Results Page}}}}

\end{itemize}




\section{Player Class}
\label{\detokenize{Player_fields:player-class}}\label{\detokenize{Player_fields::doc}}
Here’s an overview of all fields and functions in the Player class:


\subsection{Fields}
\label{\detokenize{Player_fields:fields}}

\subsubsection{Carbon}
\label{\detokenize{Player_fields:carbon}}\label{\detokenize{Player_fields:carbon-ref}}\begin{itemize}
\item {} 
The \sphinxcode{\sphinxupquote{player.carbon}} field specifies the  carbon value in lbs for each round of the CET.

\end{itemize}


\subsubsection{Car Miles}
\label{\detokenize{Player_fields:car-miles}}\label{\detokenize{Player_fields:car-miles-ref}}\begin{itemize}
\item {} 
\sphinxcode{\sphinxupquote{player.car\_miles}} field specifies the equivalent of the carbon value in miles driven in a standard car.

\item {} 
This value serves as a real\sphinxhyphen{}life example for the players to understand the carbon footprint of the decision.

\end{itemize}


\subsubsection{Bonus}
\label{\detokenize{Player_fields:bonus}}\label{\detokenize{Player_fields:bonus-ref}}\begin{itemize}
\item {} 
\sphinxcode{\sphinxupquote{player.bonus}} is a Currency field that specifies the monetary compensation a player receives,
if the environmentally unfriendly Option A is picked.

\end{itemize}

The \sphinxcode{\sphinxupquote{player.carbon}}, \sphinxcode{\sphinxupquote{player.car\_miles}} and \sphinxcode{\sphinxupquote{player.bonus}} value are initialised for all rounds before the start
of the experiment. See {\hyperref[\detokenize{Subsession_fields:creating-session-ref}]{\sphinxcrossref{\DUrole{std,std-ref}{creating\_session()}}}}


\subsubsection{Decided}
\label{\detokenize{Player_fields:decided}}\begin{itemize}
\item {} 
\sphinxcode{\sphinxupquote{player.decided}} is a Boolean field that states whether a player has made an autonomous decision or not.

\item {} 
\sphinxcode{\sphinxupquote{True}}: The player has decided on his/her own.

\item {} 
\sphinxcode{\sphinxupquote{False}}: A timeout happened.

\item {} 
A player has a \sphinxcode{\sphinxupquote{decided}} value in each round of the CET.

\end{itemize}


\subsubsection{Choice}
\label{\detokenize{Player_fields:choice}}\begin{itemize}
\item {} 
\sphinxcode{\sphinxupquote{player.choice}} specifies the choice of the player for each round.

\item {} 
\sphinxcode{\sphinxupquote{0}}: Option B: The player decides to forfeit the bonus (Environmentally friendly decision).

\item {} 
\sphinxcode{\sphinxupquote{1}}: Option A: The player decides to take the bonus (Environmentally unfriendly decision).dsfndsjfnjds

\end{itemize}


\subsubsection{Choice Practice}
\label{\detokenize{Player_fields:choice-practice}}\begin{itemize}
\item {} 
\sphinxcode{\sphinxupquote{player.choice\_practice}} is needed to produce an error if player tries to click through practice rounds.

\item {} 
This field serves no other purpose and can be disregarded in the {\hyperref[\detokenize{output:output-ref}]{\sphinxcrossref{\DUrole{std,std-ref}{Output}}}}.

\end{itemize}


\subsubsection{Total Emission}
\label{\detokenize{Player_fields:total-emission}}\begin{itemize}
\item {} 
\sphinxcode{\sphinxupquote{player.total\_emission}} is the sum of the \sphinxcode{\sphinxupquote{player.carbon}} values over all rounds the player has progressed so far.

\item {} 
This value increases every time a player progresses to a new round.

\item {} 
This field can be split into two sub\_fields: \sphinxcode{\sphinxupquote{player.chosen\_emission}} and \sphinxcode{\sphinxupquote{player.saved\_emission}}.

\end{itemize}


\subsubsection{Chosen Emission}
\label{\detokenize{Player_fields:chosen-emission}}\begin{itemize}
\item {} 
\sphinxcode{\sphinxupquote{player.chosen\_emission}} is the sum of all carbon values if Option A was clicked.

\item {} 
This value increases everytime the player chooses Option A.

\end{itemize}


\subsubsection{Saved Emission}
\label{\detokenize{Player_fields:saved-emission}}\begin{itemize}
\item {} 
\sphinxcode{\sphinxupquote{player.saved\_emission}} is the sum of all carbon values if Option B was clicked.

\item {} 
This value increases everytime the player chooses Option B.

\end{itemize}


\subsubsection{Payoff per Round}
\label{\detokenize{Player_fields:payoff-per-round}}\begin{itemize}
\item {} 
The \sphinxcode{\sphinxupquote{player.payoff\_per\_round}} field contains the player’s (hypothetical) payoff for each round.

\item {} 
If the player chooses environmentally\sphinxhyphen{}friendly Option B, the payoff is 0.

\item {} 
If the player chooses environmentally\sphinxhyphen{}unfriendly Option A, the payoff is \sphinxcode{\sphinxupquote{player.bonus}}.

\item {} 
\sphinxcode{\sphinxupquote{player.payoff\_per\_round}} is used as a helper field and does NOT define the actual \sphinxcode{\sphinxupquote{player.payoff}} field.
The actual payoff is calculated in {\hyperref[\detokenize{Player_fields:set-payoff-ref}]{\sphinxcrossref{\DUrole{std,std-ref}{Set Payoff}}}}.

\end{itemize}


\subsubsection{Is Bot}
\label{\detokenize{Player_fields:is-bot}}\begin{itemize}
\item {} 
\sphinxcode{\sphinxupquote{player.is\_bot}} is a Boolean field that states if the player is seen as a Bot or not.

\item {} 
A player is regarded as a bot if the amount of autonomous decisions falls below the {\hyperref[\detokenize{Constants:bot-criteria-ref}]{\sphinxcrossref{\DUrole{std,std-ref}{Bot Criteria}}}}.

\item {} 
The bot status of a player can vary from round to round depending on the player’s percentage of autonomous decisions.

\end{itemize}


\subsubsection{Is Dropout}
\label{\detokenize{Player_fields:is-dropout}}\label{\detokenize{Player_fields:is-dropout-ref}}\begin{itemize}
\item {} 
\sphinxcode{\sphinxupquote{player.is\_dropout}} is a Boolean field that states if the player is considered a dropout.

\item {} 
A dropout player is a player that mathematically cannot become “human” again.

\item {} 
\sphinxstylestrong{Example:}
The CET has a total of 40 rounds. Let’s assume the Bot Criteria is 0.75 so the player
has to decide autonomously in 75\% of all rounds. Hence, the player must make a decision in 30 of the total 40 rounds.
So the player can miss a total of 10 rounds and can still avoid being considered a bot if he makes a decision in
the remaining 30 rounds. However, once the player misses a total of 11 rounds, then the player can mathematically
not become “human” again. As soon as this happens the \sphinxcode{\sphinxupquote{player.is\_dropout}} field turns \sphinxcode{\sphinxupquote{True}} and the player is
considered a dropout.

\item {} 
If the player is considered a dropout, then the {\hyperref[\detokenize{pages:timeout-ref}]{\sphinxcrossref{\DUrole{std,std-ref}{Timeout}}}} for every remaining page is set to 0 seconds.
Thus, the dropout player is automatically progressed through each round until the CET is finished.
This drastically reduces the total experiment time, since experimenters don’t have to wait on dropout players.
This is espacially important because every player (also dropouts) have to finish the CET so the
\sphinxcode{\sphinxupquote{send\_payment\_mail()}} function is triggered. See {\hyperref[\detokenize{Subsession_fields:mail-call-ref}]{\sphinxcrossref{\DUrole{std,std-ref}{When is send\_payment\_mail() called?}}}}

\end{itemize}


\subsubsection{Is Finished}
\label{\detokenize{Player_fields:is-finished}}\begin{itemize}
\item {} 
\sphinxcode{\sphinxupquote{player.is\_finished}} is a Boolean field that states if the player has finished the CET or not.

\item {} 
A player is considered finished when the “Next” button on the Results page is pressed. See {\hyperref[\detokenize{pages:results}]{\sphinxcrossref{\DUrole{std,std-ref}{Results Page}}}}

\end{itemize}


\subsection{Functions}
\label{\detokenize{Player_fields:functions}}

\subsubsection{Current Question}
\label{\detokenize{Player_fields:current-question}}\begin{itemize}
\item {} 
\sphinxcode{\sphinxupquote{current\_question}} is used to help initialize \sphinxcode{\sphinxupquote{player.carbon}}, \sphinxcode{\sphinxupquote{player.car\_miles}} and \sphinxcode{\sphinxupquote{player.bonus}}.

\item {} 
This function is called in {\hyperref[\detokenize{Subsession_fields:creating-session-ref}]{\sphinxcrossref{\DUrole{std,std-ref}{creating\_session()}}}}.

\end{itemize}


\subsubsection{Set Total Emission}
\label{\detokenize{Player_fields:set-total-emission}}\begin{itemize}
\item {} 
\sphinxcode{\sphinxupquote{set\_total\_emission}} sets the \sphinxcode{\sphinxupquote{player.total\_emission}} field.

\item {} 
This function is called in every round of the CET. See {\hyperref[\detokenize{pages:exp-page}]{\sphinxcrossref{\DUrole{std,std-ref}{Experiment Page}}}}

\end{itemize}


\subsubsection{Set Chosen Emission}
\label{\detokenize{Player_fields:set-chosen-emission}}\begin{itemize}
\item {} 
\sphinxcode{\sphinxupquote{set\_chosen\_emission}} sets the \sphinxcode{\sphinxupquote{player.chosen\_emission}} field.

\item {} 
This function is called in every round of the CET. See {\hyperref[\detokenize{pages:exp-page}]{\sphinxcrossref{\DUrole{std,std-ref}{Experiment Page}}}}

\end{itemize}


\subsubsection{Set Saved Emission}
\label{\detokenize{Player_fields:set-saved-emission}}\begin{itemize}
\item {} 
\sphinxcode{\sphinxupquote{set\_saved\_emission}} sets the \sphinxcode{\sphinxupquote{player.saved\_emission}} field.

\item {} 
This function is called in every round of the CET. See {\hyperref[\detokenize{pages:exp-page}]{\sphinxcrossref{\DUrole{std,std-ref}{Experiment Page}}}}

\end{itemize}


\subsubsection{Set is Bot}
\label{\detokenize{Player_fields:set-is-bot}}\begin{itemize}
\item {} 
\sphinxcode{\sphinxupquote{set\_is\_bot}} sets the \sphinxcode{\sphinxupquote{player.is\_bot}} and the \sphinxcode{\sphinxupquote{player.is\_dropout}} field.

\item {} 
This function is called in every round of the CET. See {\hyperref[\detokenize{pages:exp-page}]{\sphinxcrossref{\DUrole{std,std-ref}{Experiment Page}}}}

\end{itemize}


\subsubsection{Set Payoff per Round}
\label{\detokenize{Player_fields:set-payoff-per-round}}\begin{itemize}
\item {} 
\sphinxcode{\sphinxupquote{set\_payoffs\_per\_round}} sets the \sphinxcode{\sphinxupquote{player.payoff\_per\_round}} field.

\item {} 
This function is called in every round of the CET. See {\hyperref[\detokenize{pages:exp-page}]{\sphinxcrossref{\DUrole{std,std-ref}{Experiment Page}}}}

\end{itemize}


\subsubsection{Set Payoff}
\label{\detokenize{Player_fields:set-payoff}}\label{\detokenize{Player_fields:set-payoff-ref}}\begin{itemize}
\item {} 
\sphinxcode{\sphinxupquote{set\_payoff}} sets the \sphinxcode{\sphinxupquote{player.payoff}} field.

\item {} 
The payoff is a built\sphinxhyphen{}in player field that does not have to be initialised and is used to determine the player’s payoff.

\item {} 
This function is dependent on the {\hyperref[\detokenize{Constants:random-payoff}]{\sphinxcrossref{\DUrole{std,std-ref}{Random Payoff}}}} constant.

\item {} 
If \sphinxcode{\sphinxupquote{random\_payoff = True}} then the \sphinxcode{\sphinxupquote{player.payoff}} = \sphinxcode{\sphinxupquote{player.payoff\_per\_round}} in the paying round and 0 in every other round.

\item {} 
Else \sphinxcode{\sphinxupquote{player.payoff}} = \sphinxcode{\sphinxupquote{player.payoff\_per\_round}} for every round.

\item {} 
This function is called in every round of the CET.

\end{itemize}




\section{Pages}
\label{\detokenize{pages:pages}}\label{\detokenize{pages:id1}}\label{\detokenize{pages::doc}}
Overview of all pages of the Carbon Emission Task


\subsection{Instruction Page}
\label{\detokenize{pages:instruction-page}}
The following instructions are presented to the player:

\noindent{\hspace*{\fill}\sphinxincludegraphics[width=700\sphinxpxdimen,height=262\sphinxpxdimen]{{Instructions}.png}\hspace*{\fill}}


\subsubsection{Timeout}
\label{\detokenize{pages:timeout}}
There is a Timeout of X seconds on this page. A timeout is needed to automatically progress dropout players.


\subsection{Practice Pages}
\label{\detokenize{pages:practice-pages}}
There is a total of three practice pages before the actual CET starts. The practice pages don’t visibly differ from
the actual experiment pages and serve the sole purpose of familiarising the participants with the task.
The following carbon, car\_miles, bonus values are used in this order for the three practice pages:
\begin{itemize}
\item {} 
0.23 lbs. CO2, 0.24 car miles, 0.2£

\item {} 
4.46 lbs. CO2, 4.47 car miles, 0.6£

\item {} 
19.85 lbs. CO2, 19.86 car miles, 1£

\end{itemize}

These values are defined in the corresponding html template of the practice page. See {\hyperref[\detokenize{templates:templates-ref}]{\sphinxcrossref{\DUrole{std,std-ref}{Templates}}}}.
No data is logged for the practice pages.

\noindent{\hspace*{\fill}\sphinxincludegraphics[width=701\sphinxpxdimen,height=317\sphinxpxdimen]{{practice1}.png}\hspace*{\fill}}


\subsubsection{Forms:}
\label{\detokenize{pages:forms}}
The following forms are used for the experiment page class:
The choice\_practice field is needed to produce an error if a player clicks the “Next” Button before choosing an Option.

\begin{sphinxVerbatim}[commandchars=\\\{\}]
\PYG{k}{class} \PYG{n+nc}{Experiment\PYGZus{}page}\PYG{p}{(}\PYG{n}{Page}\PYG{p}{)}\PYG{p}{:}
    \PYG{n}{form\PYGZus{}model} \PYG{o}{=} \PYG{l+s+s1}{\PYGZsq{}}\PYG{l+s+s1}{player}\PYG{l+s+s1}{\PYGZsq{}}
    \PYG{n}{form\PYGZus{}fields} \PYG{o}{=} \PYG{p}{[}\PYG{l+s+s1}{\PYGZsq{}}\PYG{l+s+s1}{choice\PYGZus{}practice}\PYG{l+s+s1}{\PYGZsq{}}\PYG{p}{]}  \PYG{c+c1}{\PYGZsh{} == player.choice\PYGZus{}practice}
\end{sphinxVerbatim}


\subsubsection{Timeout:}
\label{\detokenize{pages:id2}}
The timeout for practice pages is set to 20 seconds.


\subsection{Experiment Page}
\label{\detokenize{pages:experiment-page}}\label{\detokenize{pages:exp-page}}
The experiment page class is responsible for the experimental rounds of the CET.


\subsubsection{Forms}
\label{\detokenize{pages:id3}}
The following forms are used for the experiment page class:
If a player presses either Option A or B the choice is automatically logged in the \sphinxcode{\sphinxupquote{player.choice}} field.

\begin{sphinxVerbatim}[commandchars=\\\{\}]
\PYG{k}{class} \PYG{n+nc}{Experiment\PYGZus{}page}\PYG{p}{(}\PYG{n}{Page}\PYG{p}{)}\PYG{p}{:}
    \PYG{n}{form\PYGZus{}model} \PYG{o}{=} \PYG{l+s+s1}{\PYGZsq{}}\PYG{l+s+s1}{player}\PYG{l+s+s1}{\PYGZsq{}}
    \PYG{n}{form\PYGZus{}fields} \PYG{o}{=} \PYG{p}{[}\PYG{l+s+s1}{\PYGZsq{}}\PYG{l+s+s1}{choice}\PYG{l+s+s1}{\PYGZsq{}}\PYG{p}{]}  \PYG{c+c1}{\PYGZsh{} == player.choice}
\end{sphinxVerbatim}


\subsubsection{Timeout}
\label{\detokenize{pages:timeout-ref}}\label{\detokenize{pages:id4}}
In contrast to the Practice Pages’ timeout, the timeout for experiment pages is set dynamically.
Before the timeout is set, it is checked whether or not the player has dropped out of the experiment. See {\hyperref[\detokenize{Player_fields:is-dropout-ref}]{\sphinxcrossref{\DUrole{std,std-ref}{Is Dropout}}}}
\begin{itemize}
\item {} 
If \sphinxcode{\sphinxupquote{player.is\_dropout = True}}, the player has dropped out and the timeout is set to 0. All subsequent pages
are submitted instantly and the player is automatically progressed through the rest of the experiment.

\item {} 
Else the timeout is set to 20 seconds.

\end{itemize}


\subsubsection{Before\_next\_page()}
\label{\detokenize{pages:before-next-page}}
The code below is executed once a player hits the “Next” button of a given round. The order of the code below is
very important because some functions depend on results that are calculated in previous functions. This order
must not be changed unless you absolutely know what you are doing.

\begin{sphinxVerbatim}[commandchars=\\\{\}]
\PYG{k}{def} \PYG{n+nf}{before\PYGZus{}next\PYGZus{}page}\PYG{p}{(}\PYG{n+nb+bp}{self}\PYG{p}{)}\PYG{p}{:}
    \PYG{c+c1}{\PYGZsh{} Timeout check}
    \PYG{k}{if} \PYG{n+nb+bp}{self}\PYG{o}{.}\PYG{n}{timeout\PYGZus{}happened}\PYG{p}{:}
        \PYG{n+nb+bp}{self}\PYG{o}{.}\PYG{n}{player}\PYG{o}{.}\PYG{n}{decided} \PYG{o}{=} \PYG{k+kc}{False}
        \PYG{n+nb+bp}{self}\PYG{o}{.}\PYG{n}{player}\PYG{o}{.}\PYG{n}{choice} \PYG{o}{=} \PYG{l+m+mi}{0}
    \PYG{k}{else}\PYG{p}{:}
        \PYG{n+nb+bp}{self}\PYG{o}{.}\PYG{n}{player}\PYG{o}{.}\PYG{n}{decided} \PYG{o}{=} \PYG{k+kc}{True}

    \PYG{c+c1}{\PYGZsh{}Payoff functions:}
    \PYG{n+nb+bp}{self}\PYG{o}{.}\PYG{n}{player}\PYG{o}{.}\PYG{n}{set\PYGZus{}payoff\PYGZus{}per\PYGZus{}round}\PYG{p}{(}\PYG{p}{)}
    \PYG{n+nb+bp}{self}\PYG{o}{.}\PYG{n}{player}\PYG{o}{.}\PYG{n}{set\PYGZus{}payoff}\PYG{p}{(}\PYG{p}{)}

    \PYG{c+c1}{\PYGZsh{}Emission functions:}
    \PYG{n+nb+bp}{self}\PYG{o}{.}\PYG{n}{player}\PYG{o}{.}\PYG{n}{set\PYGZus{}chosen\PYGZus{}emission}\PYG{p}{(}\PYG{p}{)}
    \PYG{n+nb+bp}{self}\PYG{o}{.}\PYG{n}{player}\PYG{o}{.}\PYG{n}{set\PYGZus{}total\PYGZus{}emission}\PYG{p}{(}\PYG{p}{)}
    \PYG{n+nb+bp}{self}\PYG{o}{.}\PYG{n}{player}\PYG{o}{.}\PYG{n}{set\PYGZus{}saved\PYGZus{}emission}\PYG{p}{(}\PYG{p}{)}

    \PYG{c+c1}{\PYGZsh{}Bot check}
    \PYG{n+nb+bp}{self}\PYG{o}{.}\PYG{n}{player}\PYG{o}{.}\PYG{n}{set\PYGZus{}is\PYGZus{}bot}\PYG{p}{(}\PYG{p}{)}

    \PYG{c+c1}{\PYGZsh{}Last round check}
    \PYG{k}{if} \PYG{n+nb+bp}{self}\PYG{o}{.}\PYG{n}{round\PYGZus{}number} \PYG{o}{==} \PYG{n}{Constants}\PYG{o}{.}\PYG{n}{num\PYGZus{}rounds}\PYG{p}{:}
        \PYG{n+nb+bp}{self}\PYG{o}{.}\PYG{n}{subsession}\PYG{o}{.}\PYG{n}{set\PYGZus{}sum\PYGZus{}saved\PYGZus{}emission}\PYG{p}{(}\PYG{p}{)}

    \PYG{c+c1}{\PYGZsh{} Helpful prints}
    \PYG{n+nb+bp}{self}\PYG{o}{.}\PYG{n}{subsession}\PYG{o}{.}\PYG{n}{helpful\PYGZus{}prints}\PYG{p}{(}\PYG{p}{)}
\end{sphinxVerbatim}
\begin{description}
\item[{\sphinxstylestrong{Sequence of events (basic):}}] \leavevmode
\begin{DUlineblock}{0em}
\item[] 1. Timeout check.
\item[]
\begin{DUlineblock}{\DUlineblockindent}
\item[] \sphinxhyphen{} if a timeout happened the \sphinxcode{\sphinxupquote{player.decided}} field is set to \sphinxcode{\sphinxupquote{False}} and \sphinxcode{\sphinxupquote{player.choice}} is set to 0 (Option B)
\item[] \sphinxhyphen{} Else the \sphinxcode{\sphinxupquote{player.decided}} field is set to \sphinxcode{\sphinxupquote{True}}
\end{DUlineblock}
\item[] 2. The payoff functions are called.
\item[]
\begin{DUlineblock}{\DUlineblockindent}
\item[] \sphinxhyphen{} \sphinxcode{\sphinxupquote{set\_payoff()}} depends on \sphinxcode{\sphinxupquote{set\_payoff\_per\_round()}}
\end{DUlineblock}
\item[] 3. The emission functions are called.
\item[]
\begin{DUlineblock}{\DUlineblockindent}
\item[] \sphinxhyphen{} \sphinxcode{\sphinxupquote{set\_chosen\_emission()}} and \sphinxcode{\sphinxupquote{set\_saved\_emission()}} depend on the \sphinxcode{\sphinxupquote{player.choice}} field that is set in step 1.
\item[] \sphinxhyphen{} \sphinxcode{\sphinxupquote{set\_saved\_emssion()}} depends on \sphinxcode{\sphinxupquote{set\_total\_emission}} and \sphinxcode{\sphinxupquote{set\_chosen\_emission()}}
\end{DUlineblock}
\item[] 4. Bot check.
\item[]
\begin{DUlineblock}{\DUlineblockindent}
\item[] \sphinxhyphen{} \sphinxcode{\sphinxupquote{set\_is\_bot()}} is called and evaluates whether the player is a bot and/or a dropout.
\item[] \sphinxhyphen{} This function depends on the \sphinxcode{\sphinxupquote{player.decided}} field that is set in step 1.
\end{DUlineblock}
\item[] 5. Last round check.
\item[]
\begin{DUlineblock}{\DUlineblockindent}
\item[] \sphinxhyphen{} If a player is in the last round of the CET, \sphinxcode{\sphinxupquote{set\_sum\_saved\_emission()}} is called.
\item[] \sphinxhyphen{} This function depends on \sphinxcode{\sphinxupquote{player.choice}} (step 1), \sphinxcode{\sphinxupquote{set\_saved\_emision()}} (step 3) and \sphinxcode{\sphinxupquote{set\_is\_bot()}} (step 4).
\end{DUlineblock}
\item[] 6. Helpful prints
\item[]
\begin{DUlineblock}{\DUlineblockindent}
\item[] \sphinxhyphen{} The helpful print functions is called and the state of most player and subsession fields are printed to the terminal.
\item[] \sphinxhyphen{} This function depends on most of the above steps.
\end{DUlineblock}
\end{DUlineblock}

\end{description}


\subsection{Results Page}
\label{\detokenize{pages:results-page}}\label{\detokenize{pages:results}}
The results page is displayed after the player has finished all rounds of the CET.


\subsubsection{Timeout}
\label{\detokenize{pages:id5}}\begin{description}
\item[{The timeout logic works the same way as in the experiment pages.}] \leavevmode\begin{itemize}
\item {} 
if a player has dropped out: Timeout = 0 seconds

\item {} 
Else: Timeout = 60 seconds

\end{itemize}

\end{description}


\subsubsection{Before\_next\_page()}
\label{\detokenize{pages:id6}}
This code is executed once the player hits the “Next” button of the results page. The code below is used to send
the mail for carbon\sphinxhyphen{}emission certificate purchases.

\begin{sphinxVerbatim}[commandchars=\\\{\}]
\PYG{k}{def} \PYG{n+nf}{before\PYGZus{}next\PYGZus{}page}\PYG{p}{(}\PYG{n+nb+bp}{self}\PYG{p}{)}\PYG{p}{:}
    \PYG{c+c1}{\PYGZsh{}Is Finished fields and functions}
    \PYG{n+nb+bp}{self}\PYG{o}{.}\PYG{n}{player}\PYG{o}{.}\PYG{n}{is\PYGZus{}finished} \PYG{o}{=} \PYG{k+kc}{True}
    \PYG{n+nb+bp}{self}\PYG{o}{.}\PYG{n}{subsession}\PYG{o}{.}\PYG{n}{set\PYGZus{}all\PYGZus{}players\PYGZus{}finished}\PYG{p}{(}\PYG{p}{)}

    \PYG{c+c1}{\PYGZsh{} Helpful prints}
    \PYG{n+nb+bp}{self}\PYG{o}{.}\PYG{n}{subsession}\PYG{o}{.}\PYG{n}{helpful\PYGZus{}prints}\PYG{p}{(}\PYG{p}{)}

    \PYG{c+c1}{\PYGZsh{} All finished Check and send mail}
    \PYG{k}{if} \PYG{n+nb+bp}{self}\PYG{o}{.}\PYG{n}{subsession}\PYG{o}{.}\PYG{n}{all\PYGZus{}players\PYGZus{}finished}\PYG{p}{:}
        \PYG{n+nb+bp}{self}\PYG{o}{.}\PYG{n}{subsession}\PYG{o}{.}\PYG{n}{send\PYGZus{}payment\PYGZus{}mail}\PYG{p}{(}\PYG{n+nb+bp}{self}\PYG{o}{.}\PYG{n}{subsession}\PYG{o}{.}\PYG{n}{sum\PYGZus{}saved\PYGZus{}emission}\PYG{p}{,}
                                          \PYG{l+s+s2}{\PYGZdq{}}\PYG{l+s+s2}{lbs}\PYG{l+s+s2}{\PYGZdq{}}\PYG{p}{,}
                                          \PYG{l+s+s2}{\PYGZdq{}}\PYG{l+s+s2}{Carbon Emission Task}\PYG{l+s+s2}{\PYGZdq{}}\PYG{p}{,}
                                          \PYG{l+s+s2}{\PYGZdq{}}\PYG{l+s+s2}{John Doe}\PYG{l+s+s2}{\PYGZdq{}}\PYG{p}{,}
                                          \PYG{l+s+s2}{\PYGZdq{}}\PYG{l+s+s2}{john.doe@cet.com}\PYG{l+s+s2}{\PYGZdq{}}\PYG{p}{)}
\end{sphinxVerbatim}
\begin{description}
\item[{\sphinxstylestrong{Sequence of events:}}] \leavevmode\begin{enumerate}
\sphinxsetlistlabels{\arabic}{enumi}{enumii}{}{.}%
\item {} 
Once a player hits the “Next” button the \sphinxcode{\sphinxupquote{is\_finished}} field of the player is set to \sphinxcode{\sphinxupquote{True}}

\item {} 
The \sphinxcode{\sphinxupquote{set\_all\_players\_finished()}} function checks if every player has finished the CET.

\item {} 
Helpful information is printed to the terminal (including the number of players that have finished the CET).

\item {} 
If all players have finished the CET, the \sphinxcode{\sphinxupquote{send\_payment\_mail()}} function is called.

\end{enumerate}

\item[{\sphinxstylestrong{Mail parameters:}}] \leavevmode\begin{itemize}
\item {} 
The {\hyperref[\detokenize{Subsession_fields:sum-saved-emission-ref}]{\sphinxcrossref{\DUrole{std,std-ref}{Sum saved Emission}}}} field is the total weight of CO2 emission that was saved by participants.

\item {} 
The unit of the weight is lbs.

\item {} 
The name of the experiment is Carbon Emission Task

\item {} 
The name of the recipient is John Doe

\item {} 
the recipient’s email address is \sphinxhref{mailto:john.doe@cet.com}{john.doe@cet.com}. (Multiple addresses have to be specified in a list
e.g. {[}“\sphinxhref{mailto:john.doe@cet.com}{john.doe@cet.com}”, “\sphinxhref{mailto:jane.doe@cet.com}{jane.doe@cet.com}”{]}

\end{itemize}

\end{description}


\subsubsection{Contents of send\_payment\_mail():}
\label{\detokenize{pages:contents-of-send-payment-mail}}
This is an example of a generated email:

\noindent{\hspace*{\fill}\sphinxincludegraphics[width=700\sphinxpxdimen,height=328\sphinxpxdimen]{{email}.png}\hspace*{\fill}}

The link directs you to the donation form, where the carbon\sphinxhyphen{}emission certificate purchase can be made.
The donation form looks like this:

\noindent{\hspace*{\fill}\sphinxincludegraphics[width=1000\sphinxpxdimen,height=334\sphinxpxdimen]{{spendeformular}.png}\hspace*{\fill}}




\section{Templates}
\label{\detokenize{templates:templates}}\label{\detokenize{templates:templates-ref}}\label{\detokenize{templates::doc}}
Overview of all html templates of the Carbon Emission Task


\subsection{Instruction Page}
\label{\detokenize{templates:instruction-page}}

\subsection{Practice Pages}
\label{\detokenize{templates:practice-pages}}

\subsection{Experiment Page}
\label{\detokenize{templates:experiment-page}}

\subsection{Results Page}
\label{\detokenize{templates:results-page}}



\section{Output}
\label{\detokenize{output:output}}\label{\detokenize{output:output-ref}}\label{\detokenize{output::doc}}
This in an example of the output after participants have finished the CET.


\subsection{Instruction Page}
\label{\detokenize{output:instruction-page}}

\subsection{Practice Pages}
\label{\detokenize{output:practice-pages}}

\subsection{Experiment Page}
\label{\detokenize{output:experiment-page}}

\subsection{Results Page}
\label{\detokenize{output:results-page}}

\chapter{Indices and tables}
\label{\detokenize{index:indices-and-tables}}\begin{itemize}
\item {} 
\DUrole{xref,std,std-ref}{genindex}

\item {} 
\DUrole{xref,std,std-ref}{modindex}

\item {} 
\DUrole{xref,std,std-ref}{search}

\end{itemize}



\renewcommand{\indexname}{Index}
\printindex
\end{document}